\documentclass{standalone}
\usepackage[utf8]{inputenc}
\usepackage[ruled,linesnumbered,vlined]{algorithm2e}
\newcommand{\Q}{\mathcal{Q}}

% arara: pdflatex
% arara: latexmk: { clean: partial }
\begin{document}
    \begin{algorithm}[H]
        \label{alg:char-automata}
        \DontPrintSemicolon

        \SetKwData{t}{\textbf{true}}
        \SetKwData{f}{\textbf{false}}
        \SetKwData{not}{\textbf{not}}
        
        \SetKwData{p}{P}
        \SetKwData{Tmp}{tmp}
        \SetKwFunction{Mark}{mark}
        \SetKwFunction{ClosureO}{chiusura\(_0\)}

        inizializza la collezione \(\Q\) in modo che contenga \(P_0\) = \ClosureO{\(\{S \to S'\}\)}\;
        etichetta \(P_0\) come \texttt{unmarked}\;
        \While{\(\exists P \in \Q: P\) è \texttt{unmarked}} {
            \Mark{\(P\)}\;
            \ForEach{\(Y\) a sinistra del marker in qualche \(item\) di \(P\)} {
                calcola il kernel-set \Tmp per \(Y\) di \(P\)\;
                \eIf{\(\Q\) contiene uno stato \(Q\) di kernel \Tmp} {
                    sia \(Q\) lo \(Y\) per \(P\)\;
                }
                { 
                    aggiungi \ClosureO{\Tmp} a \(\Q\) come stato \texttt{unmarked}\;
                    sia \ClosureO{\Tmp} lo \(Y\) per \(P\)\;
                }
            }
        }

        \caption{CharAutomataConstruction(input?)}
    \end{algorithm}
\end{document}