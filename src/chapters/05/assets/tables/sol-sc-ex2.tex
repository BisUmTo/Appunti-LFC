\documentclass{standalone}

\usepackage{booktabs}
\usepackage{amsmath}
\usepackage{tabularx}
\usepackage[table]{xcolor}

\setlength\lightrulewidth{0.1pt}
\providecommand\lightrule{%
	\arrayrulecolor{black!30}%
	\midrule[\lightrulewidth]%
	\arrayrulecolor{black}}

% arara: pdflatex
% arara: latexmk: { clean: partial }
\begin{document}
\begin{tabularx}{\textwidth}{XXX}
		Stati (deterministici) & \(\varepsilon\)-chiusura dei punti di arrivo delle \(a\)-transizioni & \(\varepsilon\)-chiusura dei punti di arrivo delle \(b\)-transizioni \\
    \midrule
        Stato iniziale n.d.: \(A\) \newline
        \(T0 = \varepsilon-cl(\{A\}) = \{A, B, C, E\}\)
        &
        Stati tramite \(a\)-transizione \(\{D,E\}\) \newline
        \(\varepsilon-cl(\{D,E\}) = \{D,E\} = T1\) \newline
        [T1 unmarked]
        &
        Stati raggiunti tramite \(b\)-transizione \(\{E, A\}\) \newline
        \(\varepsilon-cl(\{A,E\}) = \{A,B,C,E\} = T0\) \newline
        [T0 è già presente!]
        \\ \lightrule
        \(T1 = \{D,E\}\)
        &
        Possibili \(a\)-tranizioni \(\{E\}\) \newline
        \(\varepsilon-cl(\{E\}) = T2\) \newline
        [T2 unmarked]
        &
        Possibili \(b\)-transizioni \(\{A,B\}\) \newline
        \(\varepsilon-cl(\{A,B\}) = \{A, B, C, E\} = T0\) \newline
        [T0 è già presente!]
        \\ \lightrule
        \(T2 = \{E\}\)
        &
        Possibili \(a\)-transizioni \(\{E\}\)
        \(\varepsilon-cl(\{E\}) = T2\)
        &
        Possibili \(b\)-transizioni \(\{A\}\)
        \(\varepsilon-cl(\{A\}) = T0\)
        \\
\end{tabularx}
\end{document}
