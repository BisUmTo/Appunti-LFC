\documentclass{standalone}

\usepackage{booktabs}
\usepackage{amsmath}
\usepackage{tabularx}
\usepackage[table]{xcolor}

\setlength\lightrulewidth{0.1pt}
\providecommand\lightrule{%
	\arrayrulecolor{black!30}%
	\midrule[\lightrulewidth]%
	\arrayrulecolor{black}}

\begin{document}
\begin{tabularx}{\textwidth}{XXX}
		Stati (determinisitci) & \(\varepsilon\)-chiusura dei punti di arrivo delle a-transizioni & \(\varepsilon\)-chiusura dei punti di arrivo delle b-transizioni \\
    \midrule
        \(T0 = \varepsilon-cl(\{1\}) = \{1,2,4\}\) \newline
        (stato iniziale) \newline
        (2 \(\implies\) stato finale)
        &
        \(T1 = \varepsilon-cl(\{3\}) = \{3\} \) \newline
        [T1 unmarked]
        &
        \(T2 = \varepsilon-cl(\{5\}) = \{5\}\) \newline 
        [T2 unmarked]
        \\ \lightrule
        \(T1 = \{3\}\) \newline
        (stato finale perché c’è 3)
        &
        \(T3 = \varepsilon-cl(\{3\}) = \{3\} = T1\) \newline
        T1 già nella collezione, non lo rimetto
        &
        \(\Phi\)
        \\ \lightrule
        \(T2 = \{5\}\) \newline
        (stato finale perché c’è 5)
        &
        \(\Phi\)
        &
        \(T4 = \varepsilon-cl(\{5\}) = \{5\} = T2\)
        T1 già nella collezione, non lo rimetto
        \\
\end{tabularx}
\end{document}
