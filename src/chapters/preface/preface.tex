\documentclass[class=book, crop=false, oneside, 12pt]{standalone}
\usepackage{standalone}
\usepackage{../../style}
% \graphicspath{{./assets/images/}}

% arara: pdflatex: { synctex: yes, shell: yes }
% arara: latexmk: { clean: partial }
\begin{document}
\chapter*{Prefazione}
\addcontentsline{toc}{chapter}{Prefazione}
% \addcontentsline{toc}{chapter}{\protect\numberline{}Prefazione}

\section*{Il progetto}
Questo testo è una dispensa di appunti scritta da studenti; lo scopo è quello di raccogliere i contenuti del corso di Linugaggi Formali e Compilatori e organizzarli secondo un'esposizione quanto più completa, efficace ed intuitiva possibile, tanto per lo studente desideroso di ottenere un'ottima padronanza degli argomenti, quanto anche per lo studente pigro in cerca di risorse per "portare a casa" l'esame.

Gli appunti sono stati presi durante il corso di Linguaggi Formali e Compilatori tenuto dalla professoressa Paola Quaglia per il Corso di Laurea in Informatica, DISI, Università degli studi di Trento, anno accademico 2020-2021. I contenuti provengono quindi primariamente dalle lezioni della professoressa, mentre invece ordine ed esposizione sono in gran parte originali. Allo stesso modo, la maggior parte degli assets (figure, grafi, tabelle, pseudocodici) sono contenutisticamente tratti dal materiale della professoressa, ma ricreati e molto spesso manipolati dagli autori; inoltre, per ottenere il risultato appena citato, è stato molto spesso necessario abbandonare quasi del tutto l'esposizione della professoressa e usarla, appunto, come canovaccio per svilupparne una originale.

\section*{Istruzioni per l'uso}
L'elaborato finale è molto lungo (circa 300 pagine); questo potrebbe allontanare alcune categorie di utenti, ma ci sentiamo di sottolineare che una tale lunghezza è supportata da numerose ragioni. La più importante è che la narrazione alterna teoria ed esercizi, in quanto questi ultimi sono spesso fondamentali a un'efficace assimilazione degli stessi concetti teorici su cui poggiano le basi. Inoltre, i passaggi di questi esercizi, nonostante molto meccanici, sono descritti nei dettagli, tanto da risultare a volte quasi ridondanti; questo potrebbe risultare fastidioso agli utenti più \emph{sgai}\footnote{svegli}, per cui consigliamo di valutare di volta in volta se studiare nel dettaglio l'ennesimo esercizio su uno stesso algoritmo, ma nondimeno raccomandiamo di non saltare nessuno di questi a piè pari. 

Preme sottolineare che la squadra non si assume alcuna responsabilità rispetto all'esito dell'esame di chi sceglie di usare questo testo come principale risorsa di studio. Nonostante noi abbiamo versato sangue per confezionare un prodotto completo, corretto e rifinito nei dettagli, non possiamo dimenticare di essere degli studenti, proni a commettere errori o malinterpretare dei passaggi come chiunque altro, e purtroppo per mancanza di tempo e mezzi non siamo riusciti a completare le 500 revisioni che ci eravamo proposti di fare. Per questo motivo dobbiamo essere intellettualmente onesti e consigliare ai lettori di fare sempre riferimento a quanto detto in classe dalla professoressa e al materiale da lei fornito e/o indicato. Per chi invece riponesse tanta fiducia in noi ad utilizzare questo testo come unica risorsa di studio, occhi aperti.

\section*{Segnalazione errori}
Se durante la lettura doveste incorrere in errori di qualsiasi tipo, tra gli altri errori di battitura, errori concettuali o di impaginazione, vi chiediamo di fare una segnalazione; ve ne saremo riconoscenti e provvederemo a correggere quanto prima. Per fare una segnalazione potete scegliere uno dei seguenti modi:
\begin{itemize}
    \item visitando la repository del progetto (la trovate qui: \url{https://github.com/filippodaniotti/Appunti-LFC}) potete trovare i nostri profili Github con relativi indirizzi email, scriveteci senza particolari pensieri;
    \item se avete le mani in pasta con \LaTeX~ potete aprire una Github issue in cui segnalate l'errore, o anche clonare la stessa repository e proporre un vostro fix con una pull request;
    \item potete contattarci tramite l'indirizzo email istituzionale, che rispetta sempre questo schema: \textit{nome.cognome@studenti.unitn.it};
    \item se ci conoscete personalmente, non esitate a contattarci via canali informali come Telegram o WhatsApp;
    \item se ci conoscete e siete a Trento potete trovare alcuni di noi in biblioteca circa ogni giorno, da oggi fino alla fine dei tempi (o della sessione).
\end{itemize}

\section*{La squadra}
Questo progetto è frutto della collaborazione di svariate personalità, ciascuna a suo modo fondamentale per il mantenimento e la riuscita dello stesso.

\begin{labeling}{sam4retas}
    \item[p!ps] Filippo Daniotti: curatore del progetto, ha curato la revisione stilistica, la creazione e il mantenimento di tabelle e pseudocodici, e ha scritto le lezioni sempre con gli appunti degli altri.
    \item[sam4retas] Samuele Conti: iniziatore e curatore del progetto, ha curato la scrittura e l'esposizione degli esercizi e la revisione e correttezza concettuale di tutto il progetto.
    \item[frab0zzo] Francesco Bozzo: tecnico, ha curato il coordinamento tecnico la scelta di pacchetti per la creazione degli assets, ha implementato la CI e fornito consulenza in merito ai problemi tecnici di latex.
    \item[f1zzo] Federico Izzo: tecnico, ha curato la creazione e il mantenimento di grafi e alberi.
    \item[t0k3n\$] Simone Alghisi, Emanuele Beozzo, Samuele Bortolotti: scrittori, hanno curato la rielaborazione e la trascrizione delle lezioni, fornendo le indicazioni sulla scelta degli assets.
    \item[ch!n4] Michele Yin: revisore, l'asia, la precisione chirurgica, si è occupato di fare una revisione ad ampio respiro sui dettagli matematici e di scrivere alcune lezioni all'occorrenza, principalmente.
    \item[j4bb] Giacomo Zanolli: revisore, si è occupato di una revisione di ampio respiro e di impostare il setup docker. 
\end{labeling}  

\section*{Nota del curatore}
Dai è stato divertente adesso vi racconto una barzelletta allora ci sono tre amici

\end{document}